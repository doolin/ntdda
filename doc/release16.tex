\documentclass{article}
\usepackage{chicago}
\newcommand{\model}[1]{{\sc #1}}
\begin{document}

\title{Release notes for Version 1.6 of DDA
for Windows NT}
\date{\today}
\author{}
\maketitle

\section*{DDA for Windows}

DDA for Windows release 1.6 is scheduled to have the following
features implemented.

%Anything added to the feature list
%costs not just implementation time, but adds to the time required 
%for debugging, to generate test cases, and actually test those cases.  


\begin{itemize}


\item File format changes to support rotation 
correction, plane stress or plane strain,
gravitational pre-loading and rockbolts.

\item Exact rotation post-correction~\cite{maclaughlin:mm97}.
For the user, rotation post-correction is implemented as a flag 
in the input file to switch between linear rotations and exact trig 
post-corrected rotations. 

\item Maclaughlin's~\citeyear{maclaughlin:mm97} gravitation phase
model.

\item Rock bolts using Gen-Hua Shi's spring model.

\item  ``Preliminary'' time dependent loading.  
This actually seems to work
as of 1.5.26, but it has not been subject to any verification 
testing whatsoever.  Propose sending this out to 
users for feedback.

%\item GL graphics with full double-buffering.

\end{itemize}


\section*{Pre-release issues}

An considerable amount of work still needs to be done to get this 
release ready to ship.   

\begin{enumerate}

\item Problems with gravity turn-on:

1. Using the same number of steps for both gravity convergence
and analysis results in the analysis not being run after the 
gravity phase runs to step limit.  

2. Contact forces will not converge if a contact length is
equal to zero.   This is probably a bug.  Fixing will require 
finding where contact lengths are set.  The problem seems to
be with the implementation of the gravity algorithm, because
analyses work otherwise.  The current trap is set at 
label {\tt c207} in function {\tt df22()}.

\item New bitmap for v1.6.

\item Gravity code is (mostly) in, and runs without 
seg faulting, but does not do anything.  That is,
the accumulated forces are not assigned to the appropriate
variables at the end of the gravity phase.

\item Issues for rock bolts:

Rock bolt endpoints are probably being updated correctly,
modulo rotation correction,
but the graphics code is not displaying the updates.  This 
is minor and should be easily fixed. 

\item Manual needs to be updated with text explaining new file
formats and screenshots of the visual editor.  The inclined 
plane problem should be written up as a tutorial style
example problem.  Users should be advised to verify that 
DDA runs the other example problems (if any) correctly,
and to submit a bug report if not.


\end{enumerate}


\section*{Unresolved problems and bugs}

Items in this list do not interfere with using DDA according 
to the programmers intentions.  Since users intentions rarely
overlap the programmers intentions, many combinations of 
interface actions will produce seg faults.  Most of these
can be fixed relatively quickly, when it becomes necessary
to do so.  In the interim, the behavior can be documented,
and fixed on user demand.

\begin{itemize}


\item For time loads, giving the same time value twice in a 
row will segfault.  This could be documented in the user
manual.  Or caught in the input code.

\item  Bad geometry file will segfault DDA.   Try and launch 
notepad to fix.


\item Anlysis files used with wrong geometry files will 
invariably segfault.  Should probably try and trap this.

\end{itemize}


\bibliographystyle{chicago}
\bibliography{dda}

\end{document}
